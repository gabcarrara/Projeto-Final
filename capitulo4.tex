\chapter{Conclusões}\label{cap:conclusoes}

Neste trabalho apresentamos uma revisão quasi-sistemática a respeito de técnicas
de negociação de requisitos de \textit{software}. Foram  identificados  33 
artigos  descrevendo  10  diferentes  técnicas de negociação de requisitos e
suas características. Essas características incluem sua descrição, o ambiente em
que as técnicas foram descritas, avaliadas ou aplicadas, os tipos de pesquisa
que vem sendo publicados, os tipos de estudo primário, os principais achados de
cada artigo e as vantagens e desvantagens reportadas para as técnicas. É possível observar que existem diferentes técnicas de negociação de requisitos, algumas não necessariamente são melhores que outras, mas sim diferentes. Sua efetividade dependerá do ambiente e do contexto em que os \textit{stakeholders} estão inseridos. A grande maioria dos artigos que encontramos abordam a técnica no ambiente acadêmico e pouco se fala na aplicação dessas técnicas na indústria.

No restante deste capítulo apresentamos as principais contribuições, limitações e as possibilidades de trabalhos futuros.

\section{Contribuições}
Entre as contribuições deste trabalho destacamos as seguintes:

\begin{itemize}
\setlength{\itemsep}{1pt}
\setlength{\itemindent}{20pt}
\item O planejamento de um protocolo de revisão quasi-sistemática que pode ser utilizado para obter uma visão ampla e imparcial a respeito de técnicas de negociação de requisitos de \textit{software}.
\item A execução do protocolo em entre Novembro de 2015 e Julho de 2016, identificando 33 artigos contendo 10 técnicas de negociação de requisitos.
\item Os resultados da revisão, que fornecem uma visão geral das técnicas de negociação de requisitos e de suas características, incluindo vantagens, desvantagens e os principais achados.
\end{itemize} 

Acreditamos que estas contribuições possam ser úteis tanto para pesquisadores quanto para praticantes. Pesquisadores podem fazer uso da visão geral das pesquisas nesta área para situar novas pesquisas.
Os praticantes, por sua vez, podem utilizar os resultados da revisão para auxiliar na escolha de uma técnica de negociação de requisitos. É importante destacar que a negociação de requisitos é uma atividade 
que ocorre na prática de forma inerente ao próprio processo de desenvolvimento de \textit{software}, seguindo alguma técnica ou não. Desta forma, os resultados aqui produzidos podem ser utilizados por praticantes na 
busca por oportunidades de melhoria em seus processos.

\section{Limitações}

Ao decorrer deste trabalho, tivemos algumas limitações, sendo as principais delas:
\begin{itemize}
\setlength{\itemsep}{1pt}
\setlength{\itemindent}{20pt}
\item A falta de conhecimento prévio sobre técnicas de negociação de requisitos de \textit{software}.
\item Dificuldades devido a não existência de taxonomia clara do que é uma técnica de negociação. Um exemplo desta dificuldade foi a utilização frequente da palavra \textit{framework}, que atrapalhou bastante o entendimento, visto que quando usada, não indicava se a utilização era referente a uma técnica ou a uma ferramenta.
\item Aplicamos somente o primeiro nível do \textit{Backward Snowballing}.
\item Não aplicamos O \textit{Forward Snowballing}. De fato, isso seria
inviável, visto que a quantidade de citações aos 33 artigos que analisamos é muito grande. Por exemplo, 412 artigos citam \cite{boehm1998using}, 191 artigos citam \cite{boehm1994software}, 93 artigos citam \cite{grunbacher2001surfacing} e 207 citam \cite{boehm1995software}, o que totaliza, somente neste conjunto de 4 artigos, 903 artigos a serem analisados.
\end{itemize} 

\section{Trabalhos Futuros}

Como trabalho futuro, dada a pouca quantidade de estudos referentes ao uso das
técnicas na indústria, propomos realizar estudos primários nesse ambiente. Um
exemplo seria uma \textit{survey} para verificar se as empresas usam as técnicas
encontradas, técnicas não encontradas ou se não usam nenhuma técnica. Caso o
estudo primário fosse um experimento ou estudo de caso, poderíamos ainda analisar alguns aspectos como a preferência dos \textit{stakeholders}, a viabilidade de se usar ferramentas e analisar se realmente são fundamentais ou se a técnica pode ser aplicada sem as mesmas; índice de satisfação com os resultados da negociação, se os \textit{stakeholders} perdem com frequência o foco ao aplicar a técnica, entre outros.
